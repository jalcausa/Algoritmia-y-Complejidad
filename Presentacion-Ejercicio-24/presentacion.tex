\documentclass{beamer}
\usetheme{Madrid}
\usecolortheme{dolphin}

\usepackage[utf8]{inputenc}
\usepackage[spanish]{babel}
\usepackage{amsmath, amsthm, amssymb}
\usepackage{tikz}
\usepackage{algorithm}
\usepackage{algpseudocode}
\usepackage{xcolor}
\usetikzlibrary{arrows,shapes,positioning}

\uselanguage{spanish}
\languagepath{spanish}
\deftranslation[to=spanish]{Theorem}{Teorema}
\deftranslation[to=spanish]{theorem}{teorema}
\deftranslation[to=spanish]{Proof}{Demostración}
\deftranslation[to=spanish]{Corollary}{Corolario}
\deftranslation[to=spanish]{Definition}{Definición}
\deftranslation[to=spanish]{Example}{Ejemplo}
\deftranslation[to=spanish]{Lemma}{Lema}

\definecolor{sccred}{RGB}{255,180,180}
\definecolor{sccblue}{RGB}{180,180,255}
\definecolor{sccgreen}{RGB}{180,255,180}

% Modificar el pie de página para mostrar solo el número de página
\setbeamertemplate{footline}{%
  \hfill\usebeamercolor[fg]{page number in head/foot}%
  \usebeamertemplate{page number in head/foot}\hspace*{2ex}\vskip2pt%
}
\setbeamertemplate{page number in head/foot}[framenumber]

\title{$2SAT \in P$}
\subtitle{Ejercicio 27 de Sipser}
\author{Juan Carlos Alcausa Luque}
\institute{Universidad de Málaga}
\date{}

\begin{document}

\begin{frame}
\titlepage
\end{frame}

\begin{frame}{Contenido}
\tableofcontents
\end{frame}

\section{Introducción}
\begin{frame}{Problema 2SAT}
\begin{block}{Definición}
Una \alert{2cnf-fórmula} es un $AND$ de cláusulas, donde cada cláusula es un $OR$ de \alert{a lo sumo dos literales}.
\end{block}

\begin{block}{Problema}
$2SAT = \{\langle \phi \rangle \mid \phi \text{ es una 2cnf-fórmula} \}$

\vspace{0.3cm}
¿Es $2SAT \in P$? Es decir, ¿existe un algoritmo polinómico para decidir si una 2cnf-fórmula es satisfacible?
\end{block}
\end{frame}

\begin{frame}{Ejemplos de Fórmulas 2-CNF}
	\begin{block}{Fórmula 1}
	$\phi_1 = (x_1 \lor x_2) \land (x_2 \lor x_3) \land (\neg x_1 \lor \neg x_3)$
	\end{block}
	
	\begin{block}{Fórmula 2}
	$\phi_2 = (x \lor y) \land (y \lor z) \land (\neg x \lor \neg z)$
	\end{block}
	
	\begin{block}{Fórmula 3}
	$\phi_3 = (x_1 \lor x_2) \land (\neg x_1 \lor x_2) \land (x_1 \lor \neg x_2) \land (\neg x_1 \lor \neg x_2)$
	\end{block}
\end{frame}

\section{Grafo de Implicaciones}
\begin{frame}{Transformación a Grafo}
\begin{block}{Equivalencias Lógicas}
Para cualquier cláusula $(a \lor b)$:
\begin{itemize}
\item $(a \lor b) \equiv (\neg a \rightarrow b)$
\item $(a \lor b) \equiv (\neg b \rightarrow a)$
\end{itemize}
\end{block}

\begin{block}{Construcción del Grafo}
\begin{itemize}
\item Por cada variable $x_i$, creamos dos nodos: $x_i$ y $\neg x_i$
\item Por cada cláusula $(a \lor b)$, añadimos dos aristas dirigidas:
    \begin{itemize}
    \item $\neg a \rightarrow b$
    \item $\neg b \rightarrow a$
    \end{itemize}
\item Para cláusulas unitarias $(a)$, añadimos $\neg a \rightarrow a$
\end{itemize}
\end{block}
\end{frame}

\begin{frame}{}
\begin{theorem}
Una fórmula 2-CNF $\phi$ es satisfacible si y solo si no existe ninguna variable $x_i$ tal que $x_i$ y $\neg x_i$ estén en la misma componente fuertemente conexa (SCC) del grafo de implicaciones.
\end{theorem}

\begin{block}{Idea}
\begin{itemize}
\item Si $x_i$ y $\neg x_i$ están en la misma SCC, entonces $x_i \Rightarrow \neg x_i$ y $\neg x_i \Rightarrow x_i$
\item Esto crea una contradicción lógica: no existe ninguna asignación válida
\item Un \alert{ciclo de inconsistencia} es un ciclo que contiene tanto $x_i$ como $\neg x_i$
\end{itemize}
\end{block}
\end{frame}

\begin{frame}{Ejemplo: Fórmula No Satisfacible}
\begin{columns}
\column{0.38\textwidth}
Consideremos:
\vspace{-0.2cm}
$$\phi_3 = \begin{array}{l}
(x \lor y) \land \\
(\neg x \lor y) \land \\
(x \lor \neg y) \land \\
(\neg x \lor \neg y)
\end{array}$$

\vspace{0.2cm}
Transformando a implicaciones:
\vspace{-0.2cm}
\begin{itemize}
\item $(\neg x \rightarrow y)$, $(\neg y \rightarrow x)$
\item $(x \rightarrow y)$, $(\neg y \rightarrow \neg x)$
\item $(\neg x \rightarrow \neg y)$, $(y \rightarrow x)$
\item $(x \rightarrow \neg y)$, $(y \rightarrow \neg x)$
\end{itemize}

\column{0.56\textwidth}
\vspace{-1.0cm}
\begin{tikzpicture}[scale=0.6, every node/.style={circle, draw, minimum size=15pt, inner sep=0pt}]
% Nodes
\node[fill=sccred] (x) at (0,1.5) {$x$};
\node[fill=sccred] (nx) at (3.5,1.5) {$\neg x$};
\node[fill=sccred] (y) at (0,-1.5) {$y$};
\node[fill=sccred] (ny) at (3.5,-1.5) {$\neg y$};

% Edges
\draw[-latex] (nx) -- (y);
\draw[-latex] (ny) -- (x);
\draw[-latex] (x) -- (y);
\draw[-latex] (ny) -- (nx);
\draw[-latex] (nx) -- (ny);
\draw[-latex] (y) -- (x);
\draw[-latex] (x) -- (ny);
\draw[-latex] (y) -- (nx);
\end{tikzpicture}

\vspace{1.3cm}
\begin{alertblock}{Resultado}
¡Todos los nodos están en la misma SCC!

Contradicción: $\phi_3$ no es satisfacible.
\end{alertblock}
\end{columns}
\end{frame}

\begin{frame}{Ejemplo: Fórmula Satisfacible}
\begin{columns}
\column{0.43\textwidth}
Consideremos:
\vspace{-0.2cm}
$$\phi_2 = \begin{array}{l}
(x \lor y) \land \\
(y \lor z) \land \\
(\neg x \lor \neg z)
\end{array}$$

\vspace{0.2cm}
Transformando a implicaciones:
\vspace{-0.2cm}
\begin{itemize}
\item $(\neg x \rightarrow y)$, $(\neg y \rightarrow x)$
\item $(\neg y \rightarrow z)$, $(\neg z \rightarrow y)$
\item $(x \rightarrow \neg z)$, $(z \rightarrow \neg x)$
\end{itemize}

\column{0.53\textwidth}
\vspace{-0.3cm}
\begin{tikzpicture}[scale=0.6, every node/.style={circle, draw, minimum size=15pt, inner sep=0pt}]
% Nodes - mantenemos la separación vertical
\node (x) at (0,2.5) {$x$};
\node (nx) at (4.0,2.5) {$\neg x$};
\node (y) at (0,0.0) {$y$};
\node (ny) at (4.0,0.0) {$\neg y$};
\node (z) at (0,-2.5) {$z$};
\node (nz) at (4.0,-2.5) {$\neg z$};

% Las 6 aristas originales (sin colorear nodos por ahora)
\draw[-latex] (nx) -- (y);
\draw[-latex] (ny) -- (x);
\draw[-latex] (ny) -- (z);
\draw[-latex] (nz) -- (y);
\draw[-latex] (x) -- (nz);
\draw[-latex] (z) -- (nx);
\end{tikzpicture}
\vspace{-0.5cm}
\end{columns}

\begin{block}{Análisis correcto}
Este grafo no tiene SCCs grandes, solo SCCs triviales (nodos individuales).
No hay ciclos que contengan tanto una variable como su negación.

Por tanto, la fórmula $\phi_2$ es satisfacible. Una asignación válida es: $x=1, y=1, z=0$.
\end{block}
\end{frame}

\section{Algoritmo y Demostración}
\begin{frame}{Demostración Formal}
\begin{block}{Teorema}
$\phi$ es satisfacible si y solo si no hay un ciclo de inconsistencia en su grafo de implicaciones.
\end{block}

\begin{proof}[Demostración]
\begin{itemize}
\item \alert{($\Rightarrow$)} Si existe un ciclo de inconsistencia, entonces $x_i \Rightarrow \neg x_i$ y $\neg x_i \Rightarrow x_i$ para algún $i$. Esto crea la equivalencia lógica $x_i \leftrightarrow \neg x_i$, que es una contradicción.

\item \alert{($\Leftarrow$)} Si no existe ciclo de inconsistencia, podemos construir una asignación satisfactoria: elegimos una variable no asignada $x_i$, le asignamos un valor (y a todos sus implicados), eliminamos esos nodos y repetimos hasta asignar todas las variables.
\end{itemize}
\end{proof}
\end{frame}

\begin{frame}{Algoritmo para 2SAT}
\begin{algorithm}[H]
\caption{2SAT (vars, cláusulas)}
\begin{algorithmic}[1]
\State grafo $\gets$ construir\_grafo\_de\_implicación(vars, cláusulas)
\State mapa\_scc $\gets$ encontrar\_SCCs(grafo)
\For{cada $x \in$ vars}
    \If{mapa\_scc[x] = mapa\_scc[-x]}
        \State \Return falso
    \EndIf
\EndFor
\State \Return verdadero
\end{algorithmic}
\end{algorithm}

\begin{block}{Explicación}
\begin{itemize}
\item Construimos el grafo de implicación
\item Encontramos todas las componentes fuertemente conexas (SCCs)
\item Verificamos que ninguna variable y su negación estén en la misma SCC
\end{itemize}
\end{block}
\end{frame}

\section{Análisis de Complejidad}
\begin{frame}{Complejidad Temporal}
\begin{equation*}
T(V, E) = O(|V| + |E| + |V| + |E| + \frac{|V|}{2} * 2) = O(|V| + |E|)
\end{equation*}

\begin{enumerate}
\item \textbf{Construcción del grafo de implicación}: $|V| + |E|$

\item \textbf{Encontrar las componentes fuertemente conexas}: $O(|V| + |E|)$
   \begin{itemize}
   \item La función \texttt{encontrar\_SCCs(grafo)} implementa el algoritmo de Kosaraju
   \item Este algoritmo tiene una complejidad temporal de $O(|V| + |E|)$
   \item \alert{Esta es la parte dominante de la complejidad total}
   \end{itemize}
   
\item \textbf{Verificación de contradicciones}: $|V|$
   \begin{itemize}
	\item El bucle \texttt{for} itera sobre cada variable, habiendo $\frac{|V|}{2}$ variables
	\item La comprobación de si una variable y su negación están en la misma SCC es una operación de tiempo constante (2 para cada iteración) gracias al mapa que devuelve el algoritmo.
   \end{itemize}
\end{enumerate}

\begin{block}{Conclusión}
El algoritmo tiene complejidad temporal $O(|V| + |E|)$, que expresado en términos de variables y cláusulas es $O(n + m)$.

Por tanto, \alert{$2SAT \in P$}
\end{block}
\end{frame}

\begin{frame}{Implicaciones Prácticas}
\begin{block}{Aplicaciones de 2SAT}
\begin{itemize}
\item \textbf{Programación lógica}: Resolución eficiente de restricciones binarias
\item \textbf{Planificación}: Problemas con restricciones mutuamente exclusivas
\item \textbf{Verificación de hardware}: Circuitos con restricciones de dos literales
\item \textbf{Toma de decisiones}: Problemas con opciones binarias y restricciones sencillas
\end{itemize}
\end{block}
\end{frame}


\begin{frame}{Conclusiones}
\begin{itemize}
\item 2SAT es un problema de satisfacibilidad resoluble en tiempo polinómico, lo que implica que $2SAT \in P$
\item La clave del enfoque es la transformación del problema a un grafo de implicaciones
\item La verificación de satisfacibilidad se reduce a comprobar la ausencia de ciclos de inconsistencia
\item El algoritmo tiene una complejidad temporal de $O(|V| + |E|)$
\end{itemize}

\end{frame}

\end{document}
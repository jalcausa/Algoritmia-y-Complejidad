\documentclass[a4paper,12pt]{article}
\usepackage[utf8]{inputenc}
\usepackage[spanish]{babel}
\usepackage{graphicx}
\usepackage{hyperref}
\usepackage{amsmath, amsthm, amssymb}
\usepackage{geometry}
\usepackage{lmodern}
\usepackage{tikz}
\usepackage{float}
\usetikzlibrary{arrows}

\newtheorem{theorem}{Teorema}
\geometry{left=3cm, right=3cm, top=3cm, bottom=3cm}

\hypersetup{
    colorlinks=true,
    linkcolor=blue,
    filecolor=magenta,
    urlcolor=cyan,
    pdftitle={Ejercicio_32},
    pdfpagemode=FullScreen,
}

\begin{document}

% Portada
\begin{titlepage}
    \pagenumbering{gobble}
    \centering
    \vspace*{\fill}
    {\Huge \bfseries Algoritmia y Complejidad\par}
    \vspace{1cm}
    {\huge Ejercicio 32\par}
    \vspace{1cm}
    {\Large Juan Carlos Alcausa Luque \par}
    \vspace*{\fill}
    {\large Universidad de Málaga \par}
\end{titlepage}
\pagenumbering{arabic}

\tableofcontents
\newpage

\section{Enunciado}
Demostrar que el lenguaje:
\[
\texttt{DOMINATING-SET} = \{ \langle G, k \rangle \mid G \text{ tiene un conjunto dominante de tamaño } k \}
\]
es NP-completo, mediante una reducción desde \texttt{VERTEX-COVER}.

\section{Definición del problema}
Un conjunto \( D \subseteq V \) es un \textbf{conjunto dominante} de un grafo no dirigido \( G = (V, E) \) si cada vértice \( v \in V \setminus D \) es adyacente a al menos un vértice en \( D \). El problema de decisión \texttt{DOMINATING-SET} consiste en determinar, dado un grafo \( G \) y un entero \( k \), si existe un conjunto dominante \( D \) tal que \( |D| \leq k \).

\section{DOMINATING-SET pertenece a NP}
El problema \texttt{DOMINATING-SET} pertenece a NP porque es posible verificar una solución candidata en tiempo polinómico. Basta con verificar que todos los vértices no pertenecientes al conjunto \( D \) están conectados a al menos un vértice en \( D \).

\begin{theorem}
\texttt{DOMINATING-SET} pertenece a NP.
\end{theorem}

\begin{proof}
Dado un grafo \( G = (V, E) \), un entero \( k \) y un subconjunto candidato \( D \subseteq V \), se puede verificar si \( D \) es un conjunto dominante en tiempo \( O(n^2) \), donde \( n = |V| \). Para cada vértice \( v \notin D \), se comprueba si existe al menos un vecino en \( D \), lo cual se puede hacer recorriendo la lista de adyacencias o la matriz de adyacencia. Como hay a lo sumo \( n \) vértices y se verifican adyacencias en \( O(n) \), la verificación es polinómica.
\end{proof}

\section{Reducción desde VERTEX-COVER}
El problema \texttt{VERTEX-COVER} es NP-completo. Dado un grafo \( G = (V, E) \) y un entero \( k \), pregunta si existe un subconjunto \( C \subseteq V \) de tamaño a lo sumo \( k \) tal que toda arista de \( G \) tiene al menos un extremo en \( C \).

\subsection*{Construcción de la reducción}
Sea \( G = (V, E) \) una instancia de \texttt{VERTEX-COVER}. Definimos una función de transformación \( f \) que construye un grafo \( G' = (V', E') \) para una instancia equivalente de \texttt{DOMINATING-SET}, tal que:
\begin{itemize}
    \item Por cada arista \( e = (u, v) \in E \), añadimos un nuevo vértice \( w_e \) y conectamos \( w_e \) con \( u \) y con \( v \).
    \item \( V' = V \cup \{ w_e \mid e \in E \} \)
    \item \( E' = \{ (u, w_e), (v, w_e) \mid e = (u, v) \in E \} \)
\end{itemize}

\begin{figure}[H]
\centering
\begin{tikzpicture}[scale=1.1]
\tikzstyle{every node}=[draw, circle, minimum size=8mm]
\node (u) at (0, 1) {$u$};
\node (v) at (2, 1) {$v$};
\node (w) at (1, 0) {$w_e$};
\draw (u) -- (w);
\draw (v) -- (w);
\end{tikzpicture}
\caption{Gadget de la reducción para cada arista \( (u,v) \)}
\end{figure}

\subsection{Ejemplo de reducción}
Supongamos que el grafo \( G \) está definido como:

\[
V = \{1, 2, 3\}, \quad E = \{(1,2), (2,3)\}
\]

Una posible cobertura de vértices de tamaño 2 sería \( C = \{2,3\} \), ya que cubre ambas aristas.

Transformamos \( G \) en \( G' \) de la siguiente forma:

\[
V' = \{1,2,3, w_{12}, w_{23} \}
\]
\[
E' = \{(1, w_{12}), (2, w_{12}), (2, w_{23}), (3, w_{23})\}
\]

Entonces, el conjunto dominante de \( G' \) también puede ser \( \{2,3\} \), ya que domina a \( w_{12} \), \( w_{23} \), y también a \( 1 \), pues está conectado con \( w_{12} \) (que está conectado a \( 1 \)).

\subsection{Equivalencia}
\subsection*{\( \Rightarrow \)}
Si \( C \subseteq V \) es un conjunto de cobertura de vértices en \( G \), entonces \( C \) es también un conjunto dominante en \( G' \). Esto es porque:
\begin{itemize}
    \item Todos los vértices originales de \( G \) están en \( V' \) y dominados si están en \( C \).
    \item Los nuevos vértices \( w_e \), añadidos para cada arista \( e = (u, v) \), están conectados a \( u \) y \( v \), al menos uno de los cuales está en \( C \) por definición de cobertura.
\end{itemize}

\subsection*{\( \Leftarrow \)}
Si \( D \subseteq V' \) es un conjunto dominante en \( G' \), entonces se puede construir un conjunto de cobertura \( C \subseteq V \) tal que:
\begin{itemize}
    \item Si algún vértice añadido \( w_e \) está en \( D \), podemos reemplazarlo por uno de sus vecinos \( u \) o \( v \) en el conjunto dominante sin perder dominancia.
    \item De este modo, conseguimos un conjunto \( C \subseteq V \) de tamaño a lo sumo \( k \), que cubre todas las aristas de \( G \).
\end{itemize}

\begin{theorem}
\texttt{VERTEX-COVER} \( \leq_p \) \texttt{DOMINATING-SET}
\end{theorem}

\begin{proof}
La transformación descrita se puede realizar en tiempo polinómico respecto al tamaño de \( G \). Además, como se ha demostrado, \( G \) tiene un \textit{vertex cover} de tamaño \( k \) si y solo si \( G' \) tiene un conjunto dominante de tamaño \( k \).
\end{proof}

\section{Preguntas}

\begin{itemize}
    \item ¿Es toda cobertura de vértices un conjunto dominante? \textbf{Sí}, ya que si cada arista está cubierta, entonces los nodos extremos están en el conjunto o son vecinos de nodos en el conjunto.
    \item ¿Es todo conjunto dominante una cobertura de vértices? \textbf{No necesariamente}. Puede ocurrir que algunos vértices estén dominados por vecinos pero que una arista no tenga ninguno de sus extremos en el conjunto.
    \item ¿Qué ocurre si el grafo tiene vértices aislados? Estos deben formar parte obligatoriamente del conjunto dominante, ya que no tienen vecinos.
    \item ¿Puede un conjunto dominante tener menor tamaño que una cobertura de vértices? En general, \textbf{sí}, aunque en el caso construido por la reducción ambos tendrán el mismo tamaño.
\end{itemize}

\section{Conclusión}
Hemos demostrado que:
\begin{itemize}
    \item \texttt{DOMINATING-SET} pertenece a NP.
    \item Existe una reducción en tiempo polinómico desde \texttt{VERTEX-COVER} a \texttt{DOMINATING-SET}.
    \item La reducción conserva la propiedad de solución entre instancias.
\end{itemize}

\textbf{Por tanto, \texttt{DOMINATING-SET} es NP-completo.}

\end{document}